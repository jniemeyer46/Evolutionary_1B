\documentclass[•]{article}
\usepackage{graphicx}
\graphicspath{ {/Users/jjniemeyer46/Desktop/Pics} }

\author{John Niemeyer\\JJNB78@mst.edu}
\title{COMP SCI 5401 FS2017 Assignment 1a}

\begin{document}
\maketitle

\section*{Problem 1}
If you compare the average final best random search result from part 1a (Problem 1) to the average final best result obtained in part 1b (Problem 1), you will see that part 1a ended up with a higher average final best random search result than part 1b's average final best result.  The average of part 1a's average final best random search result was 22.63 while the average final best result was a mere 12.33.  Quite a bit lower than part 1a's best results, which was also reflected in the t-test that was done.  The t-value for this t-test came out to be -10.77061 which is a p-value of less than .00001.  That means it would be better to look for the average final best random search result using the random search algorithm as opposed to the ($\mu$ + $\lambda$)EA.  This makes sense because the EA that was used is not at all optimized and it works a lot better for finding the average results as opposed to the best result.
\begin{center}
\noindent \includegraphics [scale=0.65] {/1bProb1}
\end{center}
\pagebreak

\section*{Problem 2}
So much like Problem 1, Problem 2's results in terms of the average final best random search results and average final best result line up about the same.  Part 1a (Problem 2) ended up with an average final best random search result of 25.5 while the average final best result for 1b was only 14.1.  The t-test ended up giving a t-value of -10.45592 which means that the p-value once again was < .00001, meaning that one could say that part 1a is better at finding the average final best random search result.

\begin{center}
\noindent \includegraphics [scale=0.65] {/1bProb2}
\end{center}
\pagebreak

\section*{Problem 3}
In the end Problem 3 was the most noticeable in terms of it's t-test value and p-value.  Part 1a's average final best random search value came out to be 50.93 while part 2a's average final best result sat at 34.87.  While these two values may look far apart still, the t-value for these was only -7.56392 with a p-value that was still less than .00001.  This was the closest match between the three problems, though there was no contest in terms of which of these algorithms is more suited to find the average final best random search result... The EA that was made is not optimized in any way, and is structure more to give the best average result over the best overall result.
\begin{center}
\noindent \includegraphics [scale=0.65] {/1bProb3}
\end{center}
\end{document}